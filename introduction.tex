\chapter{How Bitcoin Works}

\begin{summary}
The prologue of this chapter briefly explains the rationale behind writing this book. The rest of the chapter is a high-level introduction of how Bitcoin works that acts as a summary of things that the reader must know about before moving into the following chapters. The operation of the Bitcoin network is demonstrated with a walkthrough of a transaction and its journey from its creation up until its final destination, the Bitcoin blockchain.
\end{summary}

\section{Prologue}
I started teaching Bitcoin programming in 2016. I have given hundreds of presentations in meetups or seminars and from 2017 higher education courses. Every year I was improving and updating my material to keep it as relevant as possible. Luckily, Bitcoin progresses at a steady pace while always keeping backwards compatibility. This is convenient because existing material will always be valid even though better alternatives might be introduced in the future.
  
To understand the material better myself and to improve the material in my courses I started an open source Python library, called bitcoin-utils\footnote{https://github.com/karask/python-bitcoin-utils}. The library was created for education purposes and not for computational efficiency and that might be evident in certain parts of the implementation. Before I start this library I had investigated several other well-known Python libraries but I did not find an appropriate one for educational purposes. Some were too low-level with limited documentation while others where abstracting concepts that I deemed where important for students to understand.

This book is about teaching Bitcoin programming. Throughout the years I have prepared a lot of material based on my early code experiments, the bitcoin-utils library and several online resources, especially the Bitcoin Stack Exchange\footnote{https://bitcoin.stackexchange.com/} and the excellent Bitcoin Book\footnote{https://github.com/bitcoinbook/bitcoinbook} by A. Antonopoulos. While I try to always credit the initial sources that I have consulted over the years it is very possible that I have missed some. Please let me know and I will update accordingly.

I hope that this book consolidates all this teaching material with practical examples so as to help others understand Bitcoin programming.

\section{The Story of a Transaction}
\section{From Transactions to Blocks}
\section{Mining}
\section{Blocks and Nakamoto Consensus}
\section{Basic interaction with a node}

%problems? exercises? computer excercises???

